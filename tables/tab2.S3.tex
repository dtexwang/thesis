\begin{table}\centering
	
	\caption[Isotope fractionation factors used
in model calculations for microbial methane]{Isotope fractionation factors (input parameters) used
		in model calculations for microbial methane generated at 20~°C. A
		detailed description of the model setup and explanation of choices of
		fractionation factors is given in \autoref{model-of-isotopologue-systematics-during-microbial-methanogenesis}.}
	\label{tab:2:S3}
	
	\begin{threeparttable}

		
		\small
		\begin{tabular}{llll}
			\toprule
			& forward & backward & equilibrium\tabularnewline
			\midrule
			\textsuperscript{13}C/\textsuperscript{12}C isotope effect
			(\textsuperscript{13}$\alpha$) & 0.9600* & 0.9771\textsuperscript{†} &
			0.9824\textsuperscript{‡}\tabularnewline
			D/H primary isotope effect (\textsuperscript{2}$\alpha$\textsubscript{p}) &
			0.600 to 0.750\textsuperscript{§} & 0.751 to 0.939\textsuperscript{\#} &
			0.7989\textsuperscript{\textbar{}\textbar{}}\tabularnewline
			D/H secondary isotope effect (\textsuperscript{2}$\alpha$\textsubscript{s}) &
			0.8400\textsuperscript{¶} & 0.8400\textsuperscript{¶} &
			1.0000\textsuperscript{¶}\tabularnewline
			\textsuperscript{13}C-D clumped isotope effect ($\gamma$) & 0.9987 or 0.9965**
			& 0.9928 or 0.9907\textsuperscript{††} &
			1.0059\textsuperscript{‡‡}\tabularnewline
			\bottomrule
		\end{tabular}
		
		\begin{tablenotes}
			\item * From \textcite{Scheller++_2013_JACS_KIE} for the reduction of methyl-coenzyme
			M.
			
			\item † Internally-consistent value. For comparison, \textcite{Hermes++_1984_Bc}
			 reported 0.96 for formate dehydrogenase, and \textcite{Scharschmidt++_1984_Bc} reported 0.979 for alcohol dehydrogenase.
			
			\item ‡ From \textcite{Horita_2001_GCA}, who determined
			\textsuperscript{13}$\alpha$\textsubscript{\ce{CH4}/\ce{CO2}} = 0.932 at 20~°C; this
			reported value is equal to 0.9824 taken to the power of 4.
			
			\item § Free parameter. The range of values used here are similar to those
			reported for \emph{in vitro} studies of methyl-coenzyme M reductase
			(0.63 to 1.0) \parencite{Scheller++_2013_JACS_KIE} and from experimental cultures of methanogens
			(0.70 to 0.86) \parencite{Valentine++_2004_GCA}.
			
			\item \# Internally-consistent value. For comparison, \textcite{Scheller++_2013_JACS_KIE} 
			determined a value of 0.41 ± 0.04 (they reported a primary isotope effect of $k_\mathrm{H}/k_\mathrm{D}$ = 2.44 ± 0.22 for the activation of methane; the reciprocal of this value is \textsuperscript{2}$\alpha$\textsubscript{p}).
			
			\item \textbar{}\textbar{} From the value given by
			\textcite{Horibe+Craig_1995_GCA} for the equilibrium D/H fractionation factor between
			H\textsubscript{2}O(l) and CH\textsubscript{4}(g) at 20~°C.
			
			\item ¶ From \textcite{Scheller++_2013_JACS_KIE} for the reduction of methyl-CoM.  For comparison, \textcite{Roston+Kohen_2010_PNAS} 
			reported secondary D/H isotope effects associated with the reduction of an aldehyde by alcohol dehydrogenase of 0.94 for the forward reaction and 0.81 for the reverse reaction. 
			
			\item ** To fit the lowest Δ\textsuperscript{13}CH\textsubscript{3}D values we
			have observed in methanogen culture experiments (0.9987, corresponding
			to Δ\textsuperscript{13}CH\textsubscript{3}D = $-$1.3‰, \autoref{tab:2:S2}) or in
			nature (0.9965, corresponding to
			Δ\textsuperscript{13}CH\textsubscript{3}D = $-$3.5‰, \autoref{tab:2:S1}).
			Calculations for the fields shown in \mrefs[]{Figs.}{fig:2:2} and~\ref{fig:2:4} use the latter
			values. See \autoref{model-of-isotopologue-systematics-during-microbial-methanogenesis} for explanation of choice, and
			\autoref{fig:2:S5} for comparison of model results using the two different values.
			
			\item †† Internally-consistent value. For comparison, \textcite{Hermes++_1984_Bc}
			reported 0.999 for formate dehydrogenase, and \textcite{Scharschmidt++_1984_Bc} reported 0.995 for alcohol dehydrogenase.
			
			\item ‡‡ Computed equilibrium Δ\textsuperscript{13}CH\textsubscript{3}D value
			at 20~°C (\autoref{fig:2:S1}).
		\end{tablenotes}

	\end{threeparttable}
\end{table}
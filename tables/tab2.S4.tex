
\begin{table}\centering
	\begin{threeparttable}
		\caption[C\textsubscript{1}/C\textsubscript{2}, δD\textsubscript{water}, current environmental temperatures, and {[}H\textsubscript{2}{]} for sites studied]{Methane/ethane ratio, hydrogen isotopic composition of
			water, current environmental temperatures, and concentration of
			dissolved H\textsubscript{2} for sites studied. References are provided
			for previously-published descriptions of the field site; n.d., not
			determined.}
		\label{tab:2:S4}
		
		\begin{tabular}{l r r@{\hspace{0.2em}}l r@{\hspace{0.2em}}l r l}
			\toprule
			Location & C\textsubscript{1}/C\textsubscript{2}
			\textsuperscript{\textbar{}\textbar{}} & \multicolumn{2}{c}{δD\textsubscript{water}
			(‰)\textsuperscript{¶}} & \multicolumn{2}{c}{\textit{T} (°C)\textsuperscript{\#}} & {[}H\textsubscript{2}{]}** & 
			Data Sources\tabularnewline
			\midrule
			Bovine rumen, Pennsylvania, USA & n.d. & \textbf{$-$32} & ±
			10 & \textbf{39} & ± 2 & \emph{0.1--50 µM} & this
			study*\textsuperscript{,‡}, {[}1{]}\tabularnewline
			Northern Cascadia Margin sediments & \textgreater{}1000 &
			\emph{\textbf{+5}} & \emph{± 10} & \textbf{3--17} & & \emph{2--60 nM} &
			{[}2{]}\tabularnewline
			Powder River Basin, Montana, USA & \textgreater{}1000 & \textbf{$-$136} &
			± 5 & \textbf{18} & ± 2 & n.d. & this
			study\textsuperscript{§}\tabularnewline
			Cedar swamp, Massachusetts, USA & n.d. &
			\textbf{$-$21} & ± 10 & \emph{\textbf{16}} & \emph{± 5} & n.d. & this
			study\textsuperscript{‡}\tabularnewline
			Upper Mystic Lake, Mass., USA & n.d. & \textbf{$-$39} & ± 10 &
			\textbf{4} & ± 2 & n.d. & this study\textsuperscript{‡}\tabularnewline
			Lower Mystic Lake, Mass., USA & \textgreater{}1000 &
			\textbf{$-$41} & ± 10\textsuperscript{††} & \textbf{6} & ± 2 & n.d. & this
			study\textsuperscript{‡}\tabularnewline
			The Cedars, California, USA & \textgreater{}350 & \emph{\textbf{$-$37}} &
			\emph{± 10} & \emph{\textbf{17}} & \emph{± 1} & \emph{120, 310 µM} &
			{[}3{]}\tabularnewline
			CROMO, California, USA &
			\textgreater{}350 & \textbf{$-$33} & ± 10\textsuperscript{††} & \textbf{16} & ± 4 & 60--130 nM
			& this study\textsuperscript{†,‡}\tabularnewline
			Kidd Creek Mine, Ontario, Canada & 5.9--14 & \textbf{$-$34} & ± 6
			& \textbf{30} & ± 2 & \emph{0.8--8 mM} & {[}4{]}\tabularnewline
			Rebecca's Roost vent, Guaymas Basin & 140 &
			\emph{\textbf{+4}} & \emph{± 2} & \textbf{299} & ± 5 & 3.3 mM &
			{[}5{]}\tabularnewline
			Marcellus Fm., Penn., USA & 45 & \emph{\textbf{$-$44}} &
			\emph{± 10} & \textbf{51} & ± 10 & n.d. & {[}6{]}\tabularnewline
			Utica Fm., Penn., USA & 84 & \emph{\textbf{$-$40}} &
			\emph{± 15} & \textbf{93} & ± 10 & n.d. & {[}7{]}\tabularnewline
			\bottomrule
		\end{tabular}

		\begin{tablenotes}\small

			\item * H\textsubscript{2} concentrations were determined using gas
			chromatography with thermal conductivity detection at MIT. Analytical
			reproducibility is typically ±5\%.
			
			\item † H\textsubscript{2} concentrations were determined using a reduced
			gas analyzer gas chromatograph at NASA Ames \parencite{Crespo-Medina++_2014_FMicro}.
			
			\item ‡ The δD\textsubscript{water} was measured at the Boston
			University Stable Isotope Laboratory using high-temperature conversion
			gas chromatography isotope-ratio mass spectrometry. External
			reproducibility on replicate analyses of samples was ± 1--3‰ (1\emph{s},
			\emph{n} = 3--4), with the exception of cow rumen fluid (±8‰,
			1\emph{s}).
			
			\item § The δD\textsubscript{water} values were measured at the University of
			Arizona Environmental Geochemistry Laboratory via isotope-ratio mass
			spectrometry.
			
			\item \textbar{}\textbar{} Unless otherwise indicated, the
			C\textsubscript{1}/C\textsubscript{2} ratio (i.e., the ratio of the
			concentration of methane to that of ethane in a gas sample) was
			determined using gas chromatography with flame-ionization detection at
			MIT.
			
			\item ¶ The δD\textsubscript{water} values are reported with respect to the
			VSMOW scale.
			
			\item \# At some sites ambient temperatures were not directly measured
			(\emph{in italics}) and therefore were estimated; reasonable
			uncertainties on those estimates are given. At all other sites
			temperatures were measured \emph{in situ}.
			
			\item ** Dissolved \ce{H2} concentrations estimated from the literature are \emph{in italics}.
			
			\item †† At Lower Mystic Lake and CROMO, the waters in which methane was generated may have δD\textsubscript{water} values different from those in the water samples measured because of migration (see \autoref{field-notes-and-thoughts-on-selected-localities}).
			
			\item {[}1{]} Range of {[}H\textsubscript{2}{]} from \textcite{Janssen_2010_AFST}.
			
			\item {[}2{]} For the Northern Cascadia Margin samples, an average D/H ratio
			of marine sediment porewater \parencite[+5‰,][]{Friedman+Hardcastle_1988_JGR} is assumed. The
			natural variability of ±10‰ is taken as the uncertainty of this
			estimate. Downhole temperature measurements from Expedition 311 have
			been reported \parencite{IODP_x311_Proceedings}. Concentrations of H\textsubscript{2} were
			assumed to be within the range of 2--60 nM, which is typical of marine
			sediments \parencite{Lin++_2012_GCA}. The C\textsubscript{1}/C\textsubscript{2} data
			are from \textcite{Pohlman++_2009_EPSL}.
			
			\item {[}3{]} The {[}H\textsubscript{2}{]}, δD­\textsubscript{water} and
			temperature data are from \textcite{Morrill++_2013_GCA}. An uncertainty of
			±10‰ is applied to δD\textsubscript{water} to account for potential
			interannual variability. Dissolved {[}H\textsubscript{2}{]} was
			estimated from the H\textsubscript{2} mole \% in the gas phase,
			assuming equilibrium between gas bubbles and water at atmospheric pressure.
			
			\item {[}4{]} Dissolved {[}H\textsubscript{2}{]} for Kidd Creek fluids was
			estimated using gas/water flow rate data from \textcite{Holland++_2013_N}
			and gas-phase H\textsubscript{2} concentrations from \textcite{SherwoodLollar++_2008_GCA}, and assuming that all dissolved H\textsubscript{2} had
			completely partitioned into the gas phase prior to sampling. The
			C\textsubscript{1}/C\textsubscript{2} data are from \textcite{SherwoodLollar++_2002_N}.
			
			\item {[}5{]} Measured vent temperature and {[}H\textsubscript{2}{]} are from \textcite{Reeves++_2014_PNAS}, and δD\textsubscript{water} was assumed based on \textcite{Shanks++_1995_AGU-GM}.
			
			\item {[}6{]} The δD\textsubscript{water} values for formation water from the
			Marcellus Fm.\ in Pennsylvania are estimated from \textcite{Rowan++_2014_AAPGB}. Uncertainty on reservoir temperature is estimated at ±10~°C.
			
			\item {[}7{]} The δD\textsubscript{water} values for formation water from the
			Utica Fm.\ are estimated using data for Appalachian Basin brines from
			pre-Middle Devonian units presented in \textcite{Warner++_2012_PNAS}.
			Uncertainty on reservoir temperature is estimated at ±10~°C.


		\end{tablenotes}

	\end{threeparttable}
\end{table}

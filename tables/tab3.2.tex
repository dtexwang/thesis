%\begin{center}
\begin{table}[t]
	
	\centering
	
	\caption[Fluid compositions and Gibbs energy of reaction (Δ\textsubscript{r}\emph{G}) for
		abiotic methane formation at studied vent
		sites]{Fluid compositions\textsuperscript{a} used in thermodynamic calculations
		and calculated Gibbs energy of reaction (Δ\textsubscript{r}\emph{G}) for
		abiotic methane formation via \mrefs{Reaction}{eqn:3:2} at studied vent
		sites.\textsuperscript{b}}
	\label{tab:3:2}
	
	\begin{threeparttable}
		\small
		%\hspace*{-1ex}
		\begin{tabular}[]{@{} l l @{\!} r r r d{4} d{4} d{4} d{4} @{\!\!\!}}
			\toprule
%			{Field} & Vent & \emph{T} (°C)\textsuperscript{c} & \emph{P} (bar) & pH\textsuperscript{d} & $\big\sum\!$CO\textsubscript{2} (mm) & H\textsubscript{2} (mM) & CH\textsubscript{4} (mM) & Δ\textsubscript{r}\emph{G} (kJ/mol)\textsuperscript{e}\tabularnewline
			{Field} & Vent & \emph{T} (°C)\textsuperscript{c} & \emph{P} (bar) & pH\textsuperscript{d} & \multicolumn{1}{c}{$\big\sum\!$\ce{CO2} (mm)} & \multicolumn{1}{c}{\ce{H2} (mM)} & \multicolumn{1}{c}{\ce{CH4} (mM)} & \multicolumn{1}{c}{$\Delta_\text{r}G$ (kJ/mol)\textsuperscript{e}}\tabularnewline
			\midrule
			{Rainbow} & Guillaume & 361 & 230 & 3.33 & 24.3 & 16.5 & 2.13 & 
			-22\tabularnewline
			& CMSP\&P & 365 & 230 & 3.36 & 21.9 & 15.9 & 2.05 & -16\tabularnewline
			& Auberge & 370 & 230 & 3.35 & 22.8 & 15.7 & 2.16 & -11\tabularnewline
			{Von Damm} & Old Man Tree\textsuperscript{f} & 115 & 235 & 5.81 &
			1.80 & 10.5 & 1.97 & -121\tabularnewline
			& Ravelin 1\textsuperscript{f} & 145 & 235 & 5.83 & 2.52 & 13.4 & 2.02 &
			-113\tabularnewline
			& East Summit & 226 & 235 & 5.56 & 2.80 & 18.2 & 2.81 &
			-83\tabularnewline
			{Lost City} & Beehive & 96 & 70 & 10.20 & 0.18\textsuperscript{g}
			& 10.4 & 1.08 & -85\tabularnewline
			{Lucky Strike} & Medea & 270 & 170 & 3.81 & 98.0 & 0.063 & 0.89 &
			+20\tabularnewline
			& Isabel & 292 & 170 & 3.81 & 112.0 & 0.034 & 0.86 & +45\tabularnewline
			\bottomrule
		\end{tabular}
		%\hspace*{-1ex}	
		\begin{tablenotes}
			\item Analytical uncertainties (2\emph{s}) are ±2~°C for \emph{T}; ±5\% for
			H\textsubscript{2}, $\big\sum\!$CO\textsubscript{2}, and CH\textsubscript{4}; and
			±0.05 units for pH. Abbreviations: mm, mmol/kg fluid; mM, mmol/L fluid.
			
			\item \textsuperscript{a} All concentrations shown are extrapolated to
			endmember fluid composition (regressed to zero Mg content), except where
			noted. Data are from \textcite{McDermott++_2015_PNAS,Reeves++_2014_PNAS}.
			
			\item \textsuperscript{b} For each vent fluid, the energetic favorability of
			methane formation via this reaction was assessed by calculating the
			Gibbs energy of reaction (Δ\textsubscript{r}\emph{G}), defined by the
			relationship: Δ\textsubscript{r}\emph{G} = \emph{RT}
			ln(\emph{Q}/\emph{K}), where \emph{R} is the universal gas constant,
			\emph{T} is measured fluid temperature in kelvin, \emph{Q} is the
			reaction quotient, and \emph{K} is the equilibrium constant at \emph{T}
			and seafloor pressure \emph{P}. Equilibrium constants were calculated
			using thermodynamic data and standard states from \textcite{Johnson++_1992_CnG,Shock+Helgeson_1990_GCA}. For all calculations, the activity of
			H\textsubscript{2}O(\emph{l}) was assumed to be unity. Activity
			coefficients were assumed to be unity for neutral dissolved species. For
			all fluids except for that from Lost City,\textsuperscript{g} the
			concentration of CO\textsubscript{2}(\emph{aq}) was assumed to be equal
			to $\big\sum\!$CO\textsubscript{2}, a reasonable approximation given the low
			measured shipboard pH values and calculated equilibrium speciation of
			dissolved carbonate species at \emph{in situ} temperatures and seafloor
			pressures.
			
			\item \textsuperscript{c} Maximum measured vent temperature.
			
			\item \textsuperscript{d} Shipboard pH measurement (25~°C and 1 atm).
			
			\item \textsuperscript{e} A negative value of Δ\textsubscript{r}\emph{G}
			indicates a thermodynamic drive for the reaction to proceed as written
			from left to right (i.e., methane formation favored). Given
			uncertainties associated with chemical analyses and thermodynamic data,
			calculated Δ\textsubscript{r}\emph{G} values within ±5 kJ/mol of zero
			are interpreted to indicate that the reaction has approached or attained
			a state of thermodynamic equilibrium \parencite{Seewald_2001_GCA_minerals}.
			
			\item \textsuperscript{f} Concentrations for fluids from Old Man Tree and
			Ravelin 1 at Von Damm not extrapolated to zero Mg; Mg contents for these
			fluids are 14.0 and 15.0 mmol/kg fluid, respectively, indicating that
			endmember fluid has already mixed with seawater in the subsurface prior
			to discharge at the seafloor \parencite{McDermott++_2015_PNAS}.
			
			\item \textsuperscript{g} An arbitrary CO\textsubscript{2}(\emph{aq})
			concentration of 1 nmol/kg was used in thermodynamic calculations for
			the Lost City fluid, similar to \textcite{Reeves++_2014_PNAS}. The actual
			concentration value is subject to substantial uncertainty due to
			difficulties in determining the near-zero endmember $\big\sum\!$CO\textsubscript{2}
			content in vent fluids, given that some entrainment of
			$\big\sum\!$CO\textsubscript{2}-replete seawater always occurs during sampling
			\parencite{Proskurowski++_2008_S}. Varying this value by as much as ten orders
			of magnitude would not affect the conclusion that methane formation is
			thermodynamically favorable in the fluid, due to the high
			H\textsubscript{2} content and the power of 4 to which the activity of
			H\textsubscript{2}(\emph{aq}) is raised in the mass action expression.
	

		\end{tablenotes}

	\end{threeparttable}

\end{table}
%\end{center}
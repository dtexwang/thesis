\begin{abstract}
This thesis documents the origin, distribution, and fate of methane and
several of its isotopic forms on Earth.~~Using observational,
experimental, and theoretical approaches, I illustrate how the relative
abundances
of~\textsuperscript{12}CH\textsubscript{4},~\textsuperscript{13}CH\textsubscript{4},~\textsuperscript{12}CH\textsubscript{3}D,
and \textsuperscript{13}CH\textsubscript{3}D record the formation,
transport, and breakdown of methane in selected settings.

\qquad Chapter~2 reports precise determinations
of~\textsuperscript{13}CH\textsubscript{3}D, a ``clumped'' isotopologue
of methane, in samples collected from various settings representing many
of the major sources and reservoirs of methane on Earth.~~The results
show that the information encoded by the abundance
of~\textsuperscript{13}CH\textsubscript{3}D enables differentiation of
methane generated by microbial, thermogenic, and abiogenic processes.~~A
strong correlation between clumped- and hydrogen-isotope signatures in
microbial methane is identified and quantitatively linked to the
availability of H\textsubscript{2}~and the reversibility of
microbially-mediated methanogenesis in the environment.~~Determination
of~\textsuperscript{13}CH\textsubscript{3}D in combination with
hydrogen-isotope ratios of methane and water provides a sensitive
indicator of the extent of C--H bond equilibration, enables
fingerprinting of methane-generating mechanisms, and in some cases,
supplies direct constraints for locating the waters from which migrated
gases were sourced.~~Chapter~3 applies this concept to constrain the
origin of methane in hydrothermal fluids from sediment-poor vent fields
hosted in mafic and ultramafic rocks on slow- and ultraslow-spreading
mid-ocean ridges.~~The data support a hypogene model whereby methane
forms abiotically within plutonic rocks of the oceanic crust at
temperatures above ca.\ 300~$^{\circ}$C during respeciation of magmatic
volatiles, and is subsequently extracted during active, convective
hydrothermal circulation.~~Chapter~4 presents the results of culture
experiments in which methane is oxidized in the presence of
O\textsubscript{2}~by the bacterium~\emph{Methylococcus
	capsulatus}~strain Bath.~~The results show that the clumped isotopologue
abundances of partially-oxidized methane can be predicted from knowledge
of~\textsuperscript{13}C/\textsuperscript{12}C and D/H isotope
fractionation factors alone.~~

\end{abstract}